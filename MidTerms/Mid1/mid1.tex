\documentclass[11pt]{article}
\usepackage[margin=1in]{geometry}
\usepackage{amsmath,amssymb}
\usepackage{siunitx}
\usepackage{enumitem}

\setlength{\parindent}{0pt}
\setlength{\parskip}{6pt}

\begin{document}

\begin{center}
  \LARGE \textbf{PHYS 122 — Midterm Exam}\\
  \large Time: 50 minutes \hspace{1cm} Total: 100 points
\end{center}

\vspace{6pt}

\textbf{Instructions}
\begin{itemize}
  \item Show all work clearly. Unsupported answers receive little or no credit.
  \item Use SI units.
  \item Algebraic expressions should be written before numerical substitution.
  \item Clearly state assumptions where appropriate.
  \item You must draw and label diagrams (including PV diagrams) when asked.
\end{itemize}

\textbf{Equations and constants}
\begin{itemize}
  \item First Law of Thermodynamics:
  \[
    \Delta E = Q + W_{\mathrm{env}}
  \]
  \item Sign convention:  
  \(W_{\mathrm{env}} > 0\) for \textbf{compression},  
  \(W_{\mathrm{env}} < 0\) for \textbf{expansion}
  \item Ideal gas law:
  \[
    PV = nRT
  \]
  \item Work (constant pressure):
  \[
    W_{\mathrm{env}} = -P\Delta V
  \]
  \item Internal energy (ideal monatomic gas):
  \[
    E = \frac{3}{2}nRT \qquad \Delta E = \frac{3}{2}nR\Delta T
  \]
  \item Heat engine efficiency:
  \[
    e = \frac{W}{Q_h}
  \]
%   \item Carnot efficiency:
%   \[
%     \eta_C = 1 - \frac{T_c}{T_h}
%   \]
%   \item Entropy change (reservoir at constant temperature):
%   \[
%     \Delta S = \frac{Q}{T}
%   \]
%   \item Gas constant: \(R=\SI{8.314}{J\,mol^{-1}\,K^{-1}}\)
\end{itemize}

\newpage

\section*{Question 1 — Ideal gas and the first law \hfill (30 points)}
One mole of an ideal monatomic gas initially at pressure \(P_1=\SI{2.0e5}{Pa}\) and volume \(V_1=\SI{1.0e-3}{m^3}\) 
undergoes an expansion until its volume doubles and its pressure drops to  $1/2P_1$.

\begin{enumerate}[label=(\alph*)]
  \item Draw a clearly labeled PV diagram for this process. Indicate the direction of the process. \hfill (5 pts)
  \item Calculate the initial and final temperature of the gas. \hfill (5 pts)
  \item Calculate the work done on the gas \(W_{\mathrm{env}}\). State its sign and explain briefly. \hfill (5 pts)
  \item Calculate the change in internal energy \(\Delta E\). \hfill (5 pts)
  \item Calculate the heat \(Q\) transferred during the process. \hfill (5 pts)
  \item compare the root mean square speed of the gas molecules before and after the expansion. \hfill (5 pts)
\end{enumerate}

\newpage

% \section*{Question 2 — Cyclic process and PV work \hfill (35 points)}
% An ideal gas undergoes the following cycle:
% \begin{itemize}
%   \item A \(\to\) B: isothermal expansion
%   \item B \(\to\) C: isochoric cooling
%   \item C \(\to\) A: isobaric compression
% \end{itemize}
% The gas returns to its initial state A at the end of the cycle.

% \begin{enumerate}[label=(\alph*)]
%   \item Sketch a PV diagram of the complete cycle. Label all processes clearly. \hfill (10 pts)
%   \item For each step (A\(\to\)B, B\(\to\)C, C\(\to\)A), state whether the work done on the gas is positive, negative, or zero. Justify briefly. \hfill (10 pts)
%   \item Is the net work done on the gas over one full cycle positive, negative, or zero? Explain using your PV diagram. \hfill (8 pts)
%   \item What is the net change in internal energy over one complete cycle? Explain. \hfill (7 pts)
% \end{enumerate}

\newpage

% \section*{Question 3 — Heat engine and entropy \hfill (30 points)}
% A heat engine operates between a hot reservoir at \(T_h=\SI{600}{K}\) and a cold reservoir at \(T_c=\SI{300}{K}\). During each cycle, the engine absorbs \(\SI{500}{J}\) of heat from the hot reservoir.

% \begin{enumerate}[label=(\alph*)]
%   \item What is the maximum possible efficiency of this engine? Name the type of engine that achieves this efficiency. \hfill (8 pts)
%   \item For this maximum-efficiency engine, calculate the work done per cycle. \hfill (6 pts)
%   \item Calculate the heat rejected to the cold reservoir per cycle. \hfill (6 pts)
%   \item Calculate the total change in entropy of the universe per cycle. State whether this result is expected. \hfill (10 pts)
% \end{enumerate}

\vfill
\begin{center}
\textbf{End of Exam}
\end{center}

\end{document}
