\documentclass[11pt]{article}

\usepackage[margin=1in]{geometry}
\usepackage{enumitem}
\usepackage{booktabs}
\usepackage{hyperref}
\usepackage{fancyhdr}

\setlength{\parindent}{0pt}
\setlength{\parskip}{0.6em}

\pagestyle{fancy}
\fancyhf{}
\lhead{Physics Laboratory}
\rhead{Course Syllabus}
\rfoot{\thepage}

\begin{document}

\begin{center}
{\Large \textbf{Course Syllabus}}\\
\vspace{0.5em}
Physics Laboratory
\end{center}

\section*{Instructor Information}

\begin{itemize}[leftmargin=1.5em]
    \item \textbf{Instructor:} Dr.\ Reza Monadi
    \item \textbf{Office:} 52, E-14
    \item \textbf{Email:} \href{mailto:rmonadi@calpoly.edu}{rmonadi@calpoly.edu}
\end{itemize}

\section*{Course Materials}

\begin{itemize}[leftmargin=1.5em]
    \item \textbf{Textbook:} No formal textbook is required. Lab manuals will be posted weekly on Canvas under the Assignments tab.
\end{itemize}

\section*{Schedule of Experiments}

\begin{center}
\begin{tabular}{cl}
\toprule
\textbf{Week} & \textbf{Experiment} \\
\midrule
1 & Simple Pendulum \\
2 & Thermodynamics and Thermometers \\
3 & Specific Heat and Heats of Transformation \\
4 & Buoyancy and Archimedes' Principle \\
5 & Mass--Spring Oscillations \\
6 & Vibrating Strings \\
7 & Sound Resonance in Air Columns \\
8 & Diffraction and Interference of Light \\
9 & Reflection and Refraction of Light \\
10 & Thin Lenses \\
\bottomrule
\end{tabular}
\end{center}

\section*{Learning Outcomes}

By the end of this course, students will be able to:

\subsection*{Experimental Design and Data Collection}
\begin{itemize}[leftmargin=1.5em]
    \item Set up and conduct experiments in oscillations, waves, optics, and thermodynamics
    \item Use measurement instruments to collect accurate and precise data
\end{itemize}

\subsection*{Data Analysis and Interpretation}
\begin{itemize}[leftmargin=1.5em]
    \item Apply statistical and graphical methods to analyze data
    \item Perform error analysis to assess reliability
\end{itemize}

\subsection*{Application of Physical Principles}
\begin{itemize}[leftmargin=1.5em]
    \item Connect hands-on experimentation with fundamental physics concepts
    \item Compare experimental results with theoretical predictions
\end{itemize}

\subsection*{Scientific Communication}
\begin{itemize}[leftmargin=1.5em]
    \item Write clear, concise, and well-structured lab reports
    \item Present results using tables, graphs, and equations
\end{itemize}

\subsection*{Critical Thinking and Problem Solving}
\begin{itemize}[leftmargin=1.5em]
    \item Identify sources of error and propose methods to reduce them
    \item Make reasonable approximations when analyzing systems
\end{itemize}

\subsection*{Collaboration and Independent Learning}
\begin{itemize}[leftmargin=1.5em]
    \item Work effectively in teams and engage in collaborative problem-solving
    \item Strengthen self-directed learning beyond the lab manual
\end{itemize}

\subsection*{Laboratory Equipment and Technology}
\begin{itemize}[leftmargin=1.5em]
    \item Develop proficiency with oscilloscopes, thermometers, and data acquisition systems
    \item Use computational tools for data analysis and visualization when applicable
\end{itemize}

\section*{Attendance Policy}

Attendance is mandatory to receive credit. In case of emergencies, notify the instructor as soon as possible. Please arrive on time, as quizzes are administered at the beginning of each lab session.

\section*{Assessment}

\begin{itemize}[leftmargin=1.5em]
    \item \textbf{Lab Reports (70\%):}
    \begin{itemize}
        \item Data collection is performed collaboratively in groups
        \item Each group must download the lab manual and prepare a lab report
        \item Groups are assigned weekly; contact the instructor if you cannot locate your group
    \end{itemize}
    \item \textbf{Quizzes (30\%):}
    \begin{itemize}
        \item Quizzes cover material from the previous week’s experiment
        \item Expect a mix of conceptual and analytical questions
    \end{itemize}
\end{itemize}

\end{document}
