\documentclass[11pt]{article}

\usepackage[margin=1in]{geometry}
\usepackage{amsmath}
\usepackage{siunitx}

\begin{document}

\section*{Worked Example: Energy Content of Gasoline}

\textbf{Problem.}  
Suppose a person could magically convert the chemical energy of gasoline directly into useful mechanical energy.
This person drinks 0.5 liter of gasoline and bikes at a constant speed of \SI{10}{km/h} which requires an average mechanical 
power of \SI{100}{W}. How long could the cyclist ride using the energy from \textbf{0.5 liter} of gasoline? Does your answer surprise you? Why?
Gasoline has an energy content of approximately
\[
E = 3 \times 10^{7}\ \text{J/L}.
\]


\vspace{0.3in}

\hrule

\vspace{0.3in}

\textbf{Solution.}

\begin{enumerate}

\item \textbf{Energy in 0.5 liter of gasoline}

The energy content is already given per liter:
\[
E_= 0.5 \times 3 \times 10^{7}\ \text{J} = 1.5 \times 10^{7}\ \text{J}.
\]

\item \textbf{How long can we bike?}

Power is defined as energy per unit time:
\[
P = \frac{E}{t}.
\]

Solving for time,
\[
t = \frac{E}{P}.
\]

Substituting values,
\[
t = \frac{1.5 \times 10^{7}\ \text{J}}{100\ \text{J/s}}
= 1.5 \times 10^{5}\ \text{s}.
\]

Converting seconds to hours,
\[
t = \frac{1.5 \times 10^{5}}{3600} \approx 42\ \text{h}.
\]

\end{enumerate}

\textbf{Interpretation.}  

Using the energy in just \textbf{one liter of gasoline}, a cyclist traveling at \SI{15}{km/h} could ride for nearly \textbf{4 days continuously}. This illustrates how energy-dense chemical fuels are compared to human metabolic power.
The answer is surprising because we assumed an efficency of 100\% for human body. In reality, the human body is only about 20-25\% efficient at converting food energy into mechanical work, so the actual riding time would be significantly less.

\end{document}
