\documentclass[11pt]{article}

% ---------- Packages ----------
\usepackage[margin=1in]{geometry}
\usepackage[T1]{fontenc}
\usepackage{lmodern}
\usepackage{microtype}
\usepackage{enumitem}
\usepackage{booktabs}
\usepackage{tabularx}
\usepackage{hyperref}
\usepackage{xcolor}
\usepackage{graphicx}
\hypersetup{
  colorlinks=true,
  linkcolor=black,
  urlcolor=blue,
  citecolor=black
}

% ---------- Formatting helpers ----------
\setlist[itemize]{leftmargin=*, itemsep=0.25em, topsep=0.25em}
\setlength{\parindent}{0pt}
\setlength{\parskip}{0.55em}

\newcommand{\CourseTitle}{PHYS 122-07 College Physics II}
\newcommand{\Term}{Winter Quarter, 2026}
\newcommand{\InstructorName}{Reza Monadi}
\newcommand{\InstructorEmail}{\href{mailto:rmonadi@calpoly.edu}{rmondi@calpoly.edu}}

\begin{document}

% ---------- Header ----------
{\Large \textbf{Syllabus}}\\[0.2em]
{\large \textbf{\CourseTitle} \hfill \textbf{\Term}}\\

\hrule
\vspace{0.75em}

% ---------- Course info ----------
\textbf{Instructor:} \InstructorName\\
\textbf{Email:} \InstructorEmail\\
\textbf{Office:} 53-E14 \\
\textbf{Phone:} 805-756-0679\\
\textbf{Class location:} 53-202 \\
\textbf{Class time:} MWF 9:10AM-10AM

\textbf{Text:} \textit{College Physics: A Strategic Approach} by Knight, Jones, and Field, 4th edition\\
\textit{Note: we will not be using Mastering Physics or the Workbook — you only need the textbook, either in hard copy or digital format.}
\begin{center}
    \begin{figure}[h!]
        \centering
        \includegraphics[width=.2\linewidth]{book.jpg}
    \end{figure}
\end{center}

% ---------- Expectations ----------
\section*{Why do we need to learn Physics?}
Learning Physics is challenging and it demands patience, persistence, and practice. But in the end you earn something valuable. 
We are learning physics  simply to practice \textit{thinking and imagination}.  Life is full of situations that you must \textit{decide}. 
Equipped with critical thinking skills, you will make better decisions,
you will live a better life, and make other people's lives better. That's why we learn physics! 



\section*{Keys to success in this course}
\begin{itemize}
  \item Respect everyone in the class. 
    \item Communicate regularly with me and with your peers to develop our learning community. 
  \item Study the day's chapter before class and post a question/reply in the discussion board on Canvas 
  \item Attend lectures actively by taking notes effectively\footnote{Watch \href{https://www.youtube.com/watch?v=rwyCJdVnJ3I}{this video} for effective take-noting.}.
  \item Review your notes and refer to relevant textbook sections such that you can complete weekly homework assignments.
%   \item Communicate with me or your peers when you are struggling instead of suffering alone! 
   \item Complete weekly quizzes/midterms and the final exam.
   
\end{itemize}

% ---------- CLOs ----------
\section*{Course Learning Objectives}
By the end of this course you: 
\begin{enumerate}[label=(\alph*), leftmargin=*, itemsep=0.2em, topsep=0.2em]
  \item understand simple harmonic motion and recognize it around you. 
  \item understand how waves propagate in a medium.
  \item can define absolute temperature through the constant volume gas thermometer.
  \item comprehend the thermodynamic definition of the work done by a system.
  \item can apply the First Law of Thermodynamics to thermal processes.
  \item realize it is not a good idea to keep a refrigerator's door open to cool down your kitchen. 
  \item understand how different thermodynamic processes behave. 
  \item describe the workings of some simple optical instruments such as the magnifier, the microscope, the telescope, and corrective eyeglasses;
  \item recognize the wave nature of light which leads to interference and diffraction effects.
\end{enumerate}

% ---------- Grading ----------
\section*{Grade Weights}
\subsection*{Participation: required}
To receive the grade earned through assessments, you must complete at least 90\% 
of participation activities. Falling below 90\% reduces your grade proportionally. \\
Participation includes:
\begin{itemize}
    \item Submitting your work/effort on the assigned homework problems, on time. Late submissions are capped to half credit. 
    \item Participating in the Canvas discussion by posting questions and replying with good answers to your peers.
    \end{itemize}




 
\subsection*{Assessments: 100\%}
\begin{table}[h!]
\centering
\begin{tabularx}{0.78\linewidth}{@{}l r@{}}
\toprule
\textbf{Category} & \textbf{Weight} \\ \hline
\midrule
% & 10\% \\
Lab & 15\% \\
Quizzes (8) & 30\% \\
Midterms (2) & 25\% \\
% Midterm 2 & 20\% \\
% Your Highest Midterm Score & 5\% \\
Final Exam & 30\% \\
\bottomrule
Total & 100\%
\end{tabularx}
\end{table}


% ---------- Lectures ----------
\section*{Lectures}

I will not be taking attendance after Week 1, but exam scores tend to be 
correlated with attending lecture. I strongly encourage you to attend lecture regularly, take good notes, and review/complete them after class. 
Lectures are 
best opportunities for asking \textit{questions}.

% ---------- Homework ----------
\section*{Homework}
Weekly homework consists of the ``Assigned Problems'' listed in the Course 
Outline document posted on Canvas. 
Note that homework problems are in the \textit{Problems} section at the end 
of each textbook chapter, \textbf{NOT} 
the ``Questions'' section. 
Homeworks for each week are due the following Monday by 11:59pm. 
Please submit one PDF per assignment (i.e.\ compile the images of your work into one document). 

Additionally, there are ``Extra Problems'' listed in the Course Outline. 
I am happy to discuss these (or any physics problems) with you, but they will not 
be collected and will not count for extra points. 
They are for you to gain more practice working problems. 



% ---------- In-class assessments ----------
\section*{Quizzes}
Quizzes will take place during the first 20 minutes of class every Wednesday
including Week 1. Quizzes cover the preceding week’s material.\textcolor{red}{You are not allowed to use 
electronic devices including phones, tablets, AI glasses, calculators (of any kind). You only need something 
to write, draw, and think.}

Each quiz has 20 points and you will receive up to 2 points back by redoing your quiz and submitting it on Canvas. 
Your exams will be returned by the following Monday each week. Please keep them to practice with and amend 
your answers if necessary. I do not give makeup quizzes. Your final exam grade will be counted as 
your missed quiz grade, up to 2 missed quizzes. You will get a 0 for your 3rd+ missed quizzes. 

% ---------- Midterms ----------
\section*{Midterms}
Midterm 1 (Week 4) will cover material from Weeks 1--3. Midterm 2 (Week 8) will cover material from Weeks 4--7.

I do not give makeup midterms. If you miss a midterm, your other midterm will count doubly. 
If you miss both midterms, you will not pass the class. 

% ---------- Final exam ----------
\subsection*{Final Exam}
You must take the final exam to pass the course. The final exam will be cumulative, and will be similar in format to the quizzes and midterms. 

% ---------- Labs ----------
\section*{Labs}
Please refer to their lab syllabus for guidance on completing the lab portion of this course.

% ---------- Office hours ----------
\section*{Free help hours (aka Office Hours)}
This is my favorite part of teaching! 
Whether you have questions about physics, life, politics, or mountain biking, 
I am here to help you! 
If you have suggestions for me, or just need to talk,
 I am here for you. If you are overwhelmed and just need an extra pair of ears,  stop by my office 
 and I will make some tea for us. Please bring your cup/mug. 

 Attend free help hours, or schedule an appointment with me, or send me an email. It is important that you keep up; 
 if at any point you get stuck on the coursework, please connect with your peers and/or with me about it as soon as you can.

Free help  hours will be held in my office, 53-E14 unless I announce otherwise. 
  If you’d like to schedule a one-on-one meeting with me to discuss your grade, progress in the course, or a similar matter, please email me. 
 Come to free help hours, even if you don’t have questions!
% ---------- Tutoring ----------
\section*{Tutoring}
Cal Poly offers free tutoring through the Writing (Tutoring) and Learning Center. My understanding is that its scheduled tutoring appointments are reliable, but that it is much harder to do a walk-in and find adequate physics help. Please schedule your visits starting Monday of Week 1 here:
\href{https://calpoly.mywconline.com/}{calpoly.mywconline.com}.

% ---------- Health note ----------
\section*{A Note on Health}
If at any time you or your family’s mental or physical health becomes detrimental to your course participation, please contact me as soon as possible via email so that we can formulate a plan of action that best supports you. It is very important to me that we all take care of ourselves. Always follow campus guidelines around COVID-19.

% ---------- Tech assistance ----------
\section*{Technology Assistance}
If at any point you are unable to participate in the course due to technological limitations, please notify me immediately. I will do my best to help. Additionally, you may seek help at \href{https://tech.calpoly.edu}{tech.calpoly.edu}.

% ---------- Using AI ----------
\section*{Using AI}
AI has its uses, but I don’t believe they are beneficial to you in this class. 
Note ChatGPT and similar tools generate answers without understanding the physical reasoning behind them.
 Remember, we are here to think and imagine. If you let AI  think
  on your behalf,  you will miss the opportunity to practice how to think. 

% ---------- Cheating ----------
\section*{Cheating and Plagiarism}
Cheating and/or plagiarism of any kind will not be tolerated. No warnings will be given, and 
I will discipline you to the fullest extent I am allowed. 
This includes failing you for my course, as well as forwarding your name to the 
Office of Student Rights and Responsibilities. I assure you that cheating is not a good idea here and 
is not worth it at all.
% ---------- DRC ----------
\section*{Disability Support and Disability Resource Center Policy}
If you have a disability for which you are or may be requesting an accommodation, you are encouraged to contact the Disability Resource Center:
\href{https://drc.calpoly.edu/}{drc.calpoly.edu}.
I cannot honor disability accommodations other than those provided to me through the DRC.

Also note that starting this quarter, the DRC has a much stricter policy around students scheduling exams a minimum of five days in advance. Please note that due to constraints on my schedule, if you need extended time, you \textbf{MUST} schedule your exam to be taken at the DRC’s testing center. Exam start time should be as close to the in-class start time, accounting for the time it takes you to travel to the DRC’s testing center.

% ---------- Mental health ----------
\section*{Mental Health Support}
Various recent sociological studies have shown that stress, sleep problems, anxiety, depression, and alcohol use are not only increasingly common among college students, but are among the top ten health impediments to academic performance. I am always available to talk to you. Additionally, the University encourages students experiencing personal problems or situational crises to contact Counseling Services (805-756-2511) for assistance, support, and advocacy. This service is free and confidential.

% ---------- Diversity statement ----------
\section*{Diversity Statement}
I welcome anyone of any kind, with any background, and any way of thinking as long as they are willing to learn \textit{physics}. 


% ---------- Basic needs ----------
\section*{Basic Needs Support}
If at any time you find yourself in need of more food, a safer home environment, or other basic needs, 
please visit \href{https://basicneeds.calpoly.edu/}{Cal Poly Basic Needs website}.
\end{document}
