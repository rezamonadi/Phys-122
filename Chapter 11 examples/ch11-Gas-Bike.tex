\documentclass[12pt]{article}

\usepackage{amsmath}
\usepackage{siunitx}
\usepackage{geometry}
\geometry{margin=1in}

\title{Energy Estimate: Biking on One Gallon of Gasoline}
\author{}
\date{}

\begin{document}
\maketitle

\section*{Problem}
If the human body could use gasoline as an energy source, how far could a person bicycle at a constant speed of
\SI{15}{km/h} using \SI{1}{gallon} of gasoline?

\section*{Given Information}
\begin{itemize}
    \item $1~\text{gallon of gasoline} \approx 3~\text{kg} = 3000~\text{g}$
    \item Energy content of gasoline:
    \[
    1~\text{g gasoline} \approx 40~\text{kJ} = 4.0 \times 10^{4}~\text{J}
    \]
    \item Cycling power at \SI{15}{km/h}:
    \[
    P \approx 500~\text{W}
    \]
\end{itemize}

\section*{Total Available Energy}
\begin{align*}
E_{\text{in}} 
&= 3000~\text{g} \times 4.0 \times 10^{4}~\frac{\text{J}}{\text{g}} \\
&= 1.2 \times 10^{8}~\text{J}
\end{align*}

\section*{Time of Cycling}
Power is energy per unit time:
\[
P = \frac{E}{t}
\quad \Rightarrow \quad
t = \frac{E}{P}
\]

\begin{align*}
t 
&= \frac{1.2 \times 10^{8}~\text{J}}{500~\text{J/s}} \\
&= 2.4 \times 10^{5}~\text{s}
\end{align*}

Convert seconds to hours:
\begin{align*}
t 
&= \frac{2.4 \times 10^{5}}{3600} \\
&\approx 6.7 \times 10^{1}~\text{h} \approx 65~\text{h}
\end{align*}

\section*{Distance Traveled}
\begin{align*}
d 
&= v t \\
&= (15~\text{km/h})(65~\text{h}) \\
&\approx 9.8 \times 10^{2}~\text{km} \approx 1000~\text{km}
\end{align*}

\section*{Efficiency Correction}
Human mechanical efficiency is approximately
\[
e \approx 25\%
\]

Thus, the realistic distance is
\begin{align*}
d_{\text{actual}} 
&= e \, d \\
&= 0.25 \times 1000~\text{km} \\
&\approx 250~\text{km}
\end{align*}

\section*{Final Answer}
\[
\boxed{d \approx 250~\text{km}}
\]

\end{document}
