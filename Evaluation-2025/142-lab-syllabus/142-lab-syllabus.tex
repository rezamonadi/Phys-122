\documentclass[11pt]{article}

\usepackage[margin=1in]{geometry}
\usepackage{setspace}
\usepackage{enumitem}
\usepackage{booktabs}
\usepackage{fancyhdr}
\usepackage{hyperref}

\setlength{\parindent}{0pt}
\setlength{\parskip}{0.6em}

\pagestyle{fancy}
\fancyhf{}
\lhead{PHYS 142 Lab}
\rhead{Cal Poly Physics}
\rfoot{\thepage}

\begin{document}

\begin{center}
{\Large \textbf{Physics 142 Laboratory}}\\
\vspace{0.5em}
Cal Poly Physics Department
\end{center}

\section*{Basic Information}

\begin{tabular}{ll}
\textbf{Instructor:} & Dr.\ Reza Monadi \\
\textbf{Office:} & 26M-109 \\
\textbf{Email:} & \href{mailto:rmonadi@calpoly.edu}{rmonadi@calpoly.edu} \\
\textbf{Lab Location:} & 80-269 \\
\textbf{Lab Time:} & Tuesday, 12:10 PM -- 3:00 PM \\
\textbf{Textbook:} & Lab manuals posted weekly on Canvas (Assignments tab)
\end{tabular}

\section*{Schedule}

\begin{center}
\begin{tabular}{cl}
\toprule
\textbf{Week} & \textbf{Experiment} \\
\midrule
1 & Simple Pendulum \\
2 & Simple Harmonic Motion of a Spring \\
3 & Vibrating Strings \\
4 & Sound Resonance in Air Columns \\
5 & Interference and Diffraction of Light \\
6 & Refraction of Light \\
7 & Simple Lenses \\
8 & Temperature and Thermometers \\
9 & Specific Heat and Heat of Transformation \\
10 & Thermodynamic Cycles: Adiabatic Compression \\
\bottomrule
\end{tabular}
\end{center}

\section*{Learning Outcomes}

\subsection*{Experimental Design and Data Collection}
\begin{itemize}[leftmargin=1.5em]
\item Develop proficiency in setting up and conducting experiments in oscillations, waves, optics, and thermodynamics
\item Use appropriate measurement techniques and instruments for accurate data collection
\end{itemize}

\subsection*{Data Analysis and Interpretation}
\begin{itemize}[leftmargin=1.5em]
\item Apply statistical and graphical methods to analyze data
\item Use error analysis to assess reliability of results
\end{itemize}

\subsection*{Application of Physical Principles}
\begin{itemize}[leftmargin=1.5em]
\item Demonstrate understanding of fundamental concepts through experimentation
\item Compare experimental results with theoretical predictions
\end{itemize}

\subsection*{Scientific Communication}
\begin{itemize}[leftmargin=1.5em]
\item Write clear, concise, and well-structured lab reports
\item Present results using tables, graphs, and equations
\end{itemize}

\subsection*{Critical Thinking and Problem Solving}
\begin{itemize}[leftmargin=1.5em]
\item Identify sources of error and suggest improvements
\item Analyze physical systems using reasonable approximations
\end{itemize}

\subsection*{Collaborative and Independent Learning}
\begin{itemize}[leftmargin=1.5em]
\item Work effectively in teams
\item Develop self-directed learning skills beyond the lab manual
\end{itemize}

\subsection*{Laboratory Equipment and Technology}
\begin{itemize}[leftmargin=1.5em]
\item Use laboratory instruments such as oscilloscopes and thermometers
\item Apply computational tools for data analysis and visualization when appropriate
\end{itemize}

\section*{Class Attendance}

Attendance is mandatory to receive credit. In case of an emergency, contact the instructor as soon as possible.

\section*{Assessment}

\begin{itemize}[leftmargin=1.5em]
\item \textbf{Lab Reports:} Experiments are completed collaboratively. Reports are prepared individually using the provided Word-format lab manuals.
\item \textbf{Quizzes:} Weekly quizzes cover material from the previous experiment and include conceptual and analytical questions.
\item The lowest quiz score and the lowest lab report score will be dropped.
\end{itemize}

\section*{Grading}

\begin{center}
\begin{tabular}{lc}
\toprule
\textbf{Component} & \textbf{Weight} \\
\midrule
Lab Reports & 70\% \\
Quizzes & 30\% \\
\midrule
Total & 100\% \\
\bottomrule
\end{tabular}
\end{center}

\end{document}
