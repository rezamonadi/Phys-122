\documentclass[11pt]{article}

\usepackage[margin=1in]{geometry}
\usepackage{setspace}
\usepackage{enumitem}
\usepackage{booktabs}
\usepackage{hyperref}
\usepackage{fancyhdr}

\setlength{\parindent}{0pt}
\setlength{\parskip}{0.6em}

\pagestyle{fancy}
\fancyhf{}
\lhead{PHYS 143}
\rhead{Spring 2025}
\rfoot{\thepage}

\begin{document}

\begin{center}
    {\Large \textbf{PHYS 143 — Electricity and Magnetism}}\\
    \vspace{0.5em}
    Cal Poly Physics Department\\
    Spring 2025
\end{center}

\section*{Course Information}

\begin{tabular}{ll}
\textbf{Instructor:} & Dr.\ Reza Monadi \\
\textbf{Office:} & 26-M, Room 109 \\
\textbf{Email:} & \href{mailto:rmonadi@calpoly.edu}{rmonadi@calpoly.edu} \\
\textbf{Office Hours:} & Wednesdays 1--3 PM, Thursdays 3--5 PM \\
\textbf{Classroom:} & 26-104 \\
\textbf{Class Time:} & MWF 12:10 PM -- 1:00 PM
\end{tabular}

\section*{Course Description}

This is an introductory, calculus-based course on electricity and magnetism. Topics include electric charge, electric fields, electric potential, dielectrics, capacitors, current, resistance, circuits, magnetic fields, and electromagnetic induction.

\textbf{Prerequisites:} MATH 142 and PHYS 141 \\
\textbf{Recommended:} MATH 241 \\
\textbf{GE Credit:} Areas B1 and B3

\section*{Learning Outcomes}

By the end of this course, you will be able to:
\begin{itemize}[leftmargin=1.5em]
    \item Describe core concepts such as electric fields, potentials, currents, circuits, and magnetic fields
    \item Apply critical thinking and scientific communication skills
    \item Visualize and analyze electricity and magnetism problems
    \item Analyze electric fields for various charge distributions
    \item Analyze voltage and current in basic electric circuits
    \item Analyze magnetic fields produced by current configurations
\end{itemize}

\section*{Required and Recommended Materials}

\begin{itemize}[leftmargin=1.5em]
    \item \textbf{Required Textbook:} Randall D.\ Knight, \textit{Physics for Scientists and Engineers: A Strategic Approach}, 4th Edition (Pearson) \\
    ISBN: 0133942651
    \item \textbf{Optional:} OpenStax \textit{University Physics}, Volume 2 (free online)
\end{itemize}

\section*{Course Schedule}

\begin{center}
\begin{tabular}{cl}
\toprule
\textbf{Week} & \textbf{Topics} \\
\midrule
1 & Ch.\ 22: Electric Charges and Forces \\
2 & Ch.\ 23: Electric Field \\
3 & Ch.\ 24: Gauss's Law \\
4 & Review; Midterm 1; Ch.\ 25: Electric Potential \\
5 & Ch.\ 26: Potential and Electric Field \\
6 & Ch.\ 27: Current and Resistance \\
7 & Ch.\ 28: Fundamentals of Circuits \\
8 & Review; Midterm 2; Ch.\ 29: Magnetic Field \\
9 & Ch.\ 29: Magnetic Field \\
10 & Ch.\ 30: Electromagnetic Induction \\
Final & Final Exam \\
\bottomrule
\end{tabular}
\end{center}

\section*{Exams}

\begin{itemize}[leftmargin=1.5em]
    \item \textbf{Midterm 1:} Wednesday, April 23, 2025, 12:10 PM
    \item \textbf{Midterm 2:} Wednesday, May 21, 2025, 12:10 PM
    \item \textbf{Final Exam:} Monday, June 9, 2025, 10:10 AM -- 1:00 PM
\end{itemize}

All exams take place in the regular classroom.

\section*{Grading Policy}

\begin{center}
\begin{tabular}{lc}
\toprule
\textbf{Category} & \textbf{Weight} \\
\midrule
Homework & 10\% \\
Lab Reports & 15\% \\
Weekly Quizzes & 10\% \\
Midterm Exams (2) & 30\% \\
Final Exam & 35\% \\
\midrule
Total & 100\% \\
\bottomrule
\end{tabular}
\end{center}

There is no extra credit. Final letter grades are determined using the class grade distribution (z-score based). Performing above the class average is the most reliable way to earn an A or B.

\section*{Course Policies}

\begin{itemize}[leftmargin=1.5em]
    \item Attendance and participation are strongly encouraged
    \item Homework deadlines must be met
    \item Collaboration on homework is allowed, but submitted work must be your own
    \item Minimize use of phones and laptops during class
    \item Academic integrity is required; violations will be reported per university policy
\end{itemize}

\section*{Student Support Resources}

\begin{itemize}[leftmargin=1.5em]
    \item \textbf{Disability Resource Center:} Building 124, Room 119; (805) 756-1395
    \item \textbf{Learning Support Center:} Free tutoring for all Cal Poly students
    \item \textbf{Counseling and Psychological Services}
    \item \textbf{Campus Food Pantry}
\end{itemize}

\end{document}
