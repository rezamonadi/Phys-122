\documentclass[11pt]{article}

\usepackage[margin=1in]{geometry}
\usepackage{setspace}
\usepackage{enumitem}
\usepackage{titlesec}
\usepackage{hyperref}
\usepackage{fancyhdr}

\setlength{\parskip}{0.5em}
\setlength{\parindent}{0pt}

\titleformat{\section}{\large\bfseries}{}{0em}{}
\titleformat{\subsection}{\normalsize\bfseries}{}{0em}{}

\pagestyle{fancy}
\fancyhf{}
\rhead{PHYS 141}
\lhead{Syllabus}
\rfoot{\thepage}

\begin{document}

\begin{center}
    {\Large \textbf{PHYS 141 — General Physics I}}\\
    \vspace{0.5em}
    \textit{Parallel Pedagogy Syllabus}\\
    \vspace{1em}
    Cross-listed sections: PHYS 141-06 and PHYS 141-20
\end{center}

\section*{Instructor Information}
\begin{itemize}[leftmargin=1.5em]
    \item \textbf{Instructor:} Dr.\ Reza Monadi
    \item \textbf{Department:} Cal Poly Physics
    \item \textbf{Email:} \href{mailto:rmonadi@calpoly.edu}{rmonadi@calpoly.edu}
    \item \textbf{Office:} Building 52, Room E-14
\end{itemize}

\section*{Course Policies and Expectations}

\subsection*{Grading Overview}
Your grade is based on demonstrated understanding of physics concepts, not point accumulation.
Preparation---including videos, readings, homework, and projects---is essential.

\begin{itemize}[leftmargin=1.5em]
    \item \textbf{Assessments (100\%)}
    \begin{itemize}
        \item Weekly Assessments: 70\%
        \item Final Exam: 30\%
    \end{itemize}
    \item \textbf{Participation (Required)}
    \begin{itemize}
        \item At least 90\% completion required to receive the earned assessment grade
        \item Falling below 90\% reduces the final grade
    \end{itemize}
\end{itemize}

\subsection*{Grade Replacement Policy}
The final exam is cumulative. If a final-exam question score exceeds the corresponding weekly
assessment score, the higher score replaces the lower one. This also functions as a built-in makeup
policy. Please do not request assessments outside regular class time.

\section*{Assessments}

\subsection*{Weekly Assessments}
\begin{itemize}[leftmargin=1.5em]
    \item Held Wednesdays (approximately 20 minutes)
    \item First two weeks do not count toward the final grade
\end{itemize}

\subsection*{Final Exam}
\begin{itemize}[leftmargin=1.5em]
    \item Cumulative
    \item Weighted 30\%
    \item Functions as grade replacement
\end{itemize}

\subsection*{Assessment Rules}
\begin{itemize}[leftmargin=1.5em]
    \item No calculators permitted
    \item Focus on concepts, diagrams, and symbolic reasoning
    \item Numerical answers within $\pm 20\%$ are acceptable
\end{itemize}

\subsection*{Assessment Rubric}
\begin{description}[leftmargin=2em]
    \item[A (4/4)] Complete solution with clear statement, diagram, method, correct answer with units, and sense check
    \item[B (3/4)] Clear statement, diagram, and valid method
    \item[C (2/4)] Correct statement and labeled physics diagram
    \item[D (1/4)] Statement and minimal diagram identifying the concept
    \item[F (0/4)] Does not meet minimum threshold
\end{description}

\textit{Note: A score of 2/4 (C) is considered passing per university GPA policy.}

\section*{Participation Requirements}
To qualify for full assessment-based grading, at least 90\% of participation activities must be completed.

\subsection*{Videos}
\begin{itemize}[leftmargin=1.5em]
    \item Watch assigned videos before class
    \item Embedded questions required
    \item Full credit for on-time completion; half credit if late
    \item Final grade penalty equals twice the percentage of videos missed
\end{itemize}

\subsection*{Readings}
Verified through short reading quizzes at the beginning of class.

\subsection*{Homework}
\begin{itemize}[leftmargin=1.5em]
    \item Due Tuesdays before class
    \item Counts as two videos per week
    \item Not graded for correctness
\end{itemize}

\subsection*{Projects}
\begin{itemize}[leftmargin=1.5em]
    \item Project 1 (Week 3): Video analysis of motion (human power production)
    \item Project 2 (Week 7): Data analysis using the PhyPhox app
    \item Two students per group
\end{itemize}

\subsection*{Surveys}
Count as video participation.

\subsection*{Idea Sheets}
Students may bring a personal idea sheet (maximum 50 items, diagrams included). Excessive information
may require removal during an exam.

\section*{Course Materials}
\begin{itemize}[leftmargin=1.5em]
    \item \textbf{Textbook/Workbook:} \textit{Mechanics in Parallel}
    \item \textbf{Notebook:} Required for drawings and notes (“thinking with your hands”)
\end{itemize}

\section*{Policies}

\subsection*{Attendance and Illness}
If you are sick (including COVID-like symptoms), please stay home. Stay connected by having group
members Zoom you into class if needed.

\subsection*{24-Hour Policies}
\begin{enumerate}[leftmargin=1.5em]
    \item Class prep requirements will not change within 24 hours of class
    \item Assessments will be returned within 24 hours (by Thursday)
    \item Results will not be discussed until 24 hours after return
\end{enumerate}

\subsection*{Cell Phones}
Phones should be put away unless used for class experiments or remote group participation.

\subsection*{Cheating}
Suspected cheating results in an F and a report per university policy. If you are struggling,
please talk with the instructor before making poor decisions.

\section*{Classroom Values}
\begin{itemize}[leftmargin=1.5em]
    \item Commitment to student learning
    \item Preparation and mutual respect
    \item Collaboration over competition
    \item Diversity, inclusion, empathy, and free inquiry
\end{itemize}

\section*{Student Learning Outcomes}
By the end of the course, students will be able to:
\begin{enumerate}[leftmargin=1.5em]
    \item Explain the four analytical lenses of mechanics
    \item Select and justify appropriate modeling approaches
    \item Design and conduct simple experiments
    \item Apply scientific reasoning to complex problems
    \item Collaborate on evidence-based explanations
    \item Recognize limits of current understanding
    \item Reflect on local and global impacts of actions
    \item Articulate the role of empathy in science
\end{enumerate}

\end{document}
