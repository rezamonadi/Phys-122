\documentclass[12pt]{article}

\usepackage[margin=1in]{geometry}
\usepackage{amsmath,amssymb}
\usepackage{physics}
\usepackage{enumitem}
\usepackage{fancyhdr}
\usepackage{setspace}

% ---------- Header / Footer ----------
\pagestyle{fancy}
\fancyhf{}
\lhead{PHYS 122 Midterm Exam 1}
\rhead{Winter 2026, Monadi}
\cfoot{\thepage}

% ---------- Custom Commands ----------
\newcommand{\pts}[1]{\hfill \textbf{[#1 points]}}

\begin{document}

\begin{center}
    {\Large \textbf{PHYS 122 Midterm Exam}} \\[1ex]
    % {\large Thermal Physics} \\[2ex]
    \textbf{Time: 50 minutes} \\
    \textbf{Total Points: 100}
\end{center}

\vspace{1em}

\noindent
\textbf{Full Name:} \rule{3.5in}{0.4pt} \hfill
% \textbf{Student ID:} \rule{2in}{0.4pt}

\vspace{1.5em}

% ---------- Instructions ----------
\section*{Instructions}
\begin{itemize}
    \item Show all work clearly. Answers with no justification may receive little or no credit.
    \item Diagrams (including $PV$ diagrams) must be drawn by the student when appropriate.
\end{itemize}

\vspace{1em}

% ---------- Equation Sheet ----------
\section*{Useful Equations}
\begin{itemize}[itemsep=0.4em]
    \item \quad $PV = nRT$
    \item\quad $W_{\text{env}} = -\displaystyle\int P\,dV$ 
    \item \quad $\Delta E_{\text{th}} = Q + W_{\text{env}}$
    % \item Heat engine efficiency: \quad $e = \dfrac{W_{\text{out}}}{Q_H}$
    \item \quad $\Delta E_{\text{th}} = 3/2n R\Delta T$
    \item $v_{rms} = \sqrt{\frac{3K_BT}{m}}$
    \item $\Delta L = \alpha L \Delta L$
    \item $\Delta V = \beta V \Delta V$
\end{itemize}

\newpage

% ---------- Question 1 ----------
\section*{Question 1 \pts{60}}
A sample of an ideal gas undergoes a thermodynamic process. It starts with a Volume of 0.1 $m^{3}$ and a pressure of 
1.0 kPa and then both of its vulme and pressure get doubled. 

\begin{enumerate}[label=(\alph*)]
    \item Draw a P-V diagram, with labels and units. 
    \item Calculate the work of the environment during the process.
    \item Find the initial and final tempreture of the gas.
    \item Determine the change in thermal energy of the gas.
    \item How much heat was transfered to or from the gas?
    \item What is the ratio of final to initial root mean sqaure velocities of the gas?
\end{enumerate}

\newpage

% ---------- Question 3 ----------
\section*{Question 2 \pts{20}}
An 100 m aluminum rod is cooled down from 220 C to 120 C. The linear thermal expansion of the aluminum is 
$2.3\times10^{-5} C^{-1}$ and the volume expansion coefficent is $6.9\times10^{-5} C^{-1}$.
\begin{enumerate}[label=(\alph*)]
    \item What is the new length of the rod?
    \item Explain why the length of rod changes. 
    \end{enumerate}

\newpage

% ---------- Question 4 ----------
\section*{Question 3 \pts{20}}
You have a container and put some ice and hot water in it. 
You close its door and cover it with an insulator and wait for 1 hour. Then you open it and see 
there is more ice and some hot steam in the container. Will you get 
surprized or you think it's possible? Explain. 

\end{document}
