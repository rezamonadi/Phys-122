\documentclass[11pt]{article}
\usepackage{amsmath,amssymb}
\usepackage{geometry}
\usepackage{enumitem}
\usepackage{graphicx}
\usepackage{siunitx}
\geometry{margin=1in}

\title{Preliminary Math Self-Assessment\\Physics 141}
\author{Jacks}
\date{Fall 2023}

\begin{document}
\maketitle

\noindent The following self-assessment is an opportunity for you to determine if you should review some basic math
 skills that are essential for success in Introductory Physics I. Please complete the questions below 
 without using any resources expect or a calculator. Estimate your answers when you need to.  You do not need to time yourself. I will not collect this assessment, but I expect a hard copy of your completed assessment to be in your class binder throughout the quarter.

\medskip
\noindent Use the space below each problem to show your work clearly; showing your work clearly is essential to your success in this class. Further, if you make an error, showing your work clearly allows you to find your error and correct your understanding more efficiently.

\medskip
\noindent \textbf{ONLY AFTER} you’ve completed the assessment, compare your answers to the solution key provided. Questions that you missed should be followed up by reviewing relevant concepts and practicing problems in any algebra or pre-calculus textbook. Reading my solutions and convincing yourself you now understand how to solve the problem does not qualify as reviewing the material.

\medskip
\noindent Note that an understanding of physics is not necessary to answer any of the following questions.

\begin{enumerate}[label=\arabic*.]
\item $(4^3)(4^5)=$

\item A circle’s area is given by $A = \pi r^2$, where $r$ is the radius. Two circles have different radii, $r_1$ and $r_2$, with corresponding areas $A_1$ and $A_2$. Find $A_2$ if $A_1 = 6~\text{m}^2$ and $r_2 = 2r_1$.

\item $\frac{30}{4}\times\frac{2}{5}=$   

\item Several measurements are made of students’ heights. What is the average height? Measured heights (in meters): 1.6, 1.8, 1.75, 2.1, 1.65, 1.8, 1.9.

\item An angle is measured to be $2\frac{1}{4}$ radians. Find $\cos(\theta)$.
\newpage

\item A relationship between position and time, $x(t)=bt^2$, is shown in the scatter plot. Estimate $b$ from the data. Do not forget units.
\begin{center}
    \begin{figure}[h!]
        \centering
       \includegraphics[width=0.7\textwidth]{scatterPlot.png}
    \end{figure}

\end{center}    


\item How many seconds are in one year?

\item Express your results for the following in scientific notation:

 $\frac{(2.3\times10^{20})(3.8\times10^{-15})}{(8\times10^8)}$



\item Evaluate $e^{-2.4}$.
\newpage
\item For the picture shown, find $\phi$ and $\alpha$ in terms of $\theta$.
\begin{center}
    \begin{figure}[h!]
        \centering
       \includegraphics[width=0.5\textwidth]{angel.png}
    \end{figure}
\end{center}

\item Find $\ln(f)$ given $f = I/I_0$ and $I = 6I_0$.


\item Evaluate $\frac{1}{2}+\frac{1}{5}$.


\item Sketch $y(t)=3t+2$, where $y$ is in meters and $t$ is in seconds. Be sure your axes include scales, labels, and units.



\item Find $F$ and $a$ if $F-6a=20$ and $-F+8a=0$.

\item Evaluate $\frac{2}{5}+\frac{7}{10}$.

\item $\frac{1}{2} + \frac{1}{5}=$

\item Find $a/b+c/d$ when $b=3d$, $a=6c$, and $c=2d$.

\item Given $\frac{1}{f}=\frac{1}{s}+\frac{1}{d}$, solve for $d$ in terms of $f$ and $s$.


\item For the triangle shown, $a=3~\text{m}$ and $c=5~\text{m}$. Find $b$, $\sin(\theta)$, and $\tan(\theta)$.


\item Given $F_yL-MgL/2+(2M)gL/3=0$, find an expression for $F_y$ in terms of $M$, $g$, and $L$.

\item Given $\dfrac{GMm}{x^2}=\dfrac{GMm}{(d-x)^2}$, determine an expression for $x$ in terms of $M$, $m$, $G$, and $d$.

\item A drawer contains 9 red socks and 11 blue socks. It is dark and you are trying to grab a matching pair. How many socks do you need to take out to be sure of a matching pair?

\item For certain charges, the electric field is given by $E(x)=b/x^2$. At the location $x_0$, $E(x_0)=E_0$. Find $E(x)$ for $x=4x_0$ in terms of $E_0$.

\item Given $2I_1+2I_2-5=0$ and $I_1-4I_2=0$, find $I_1$ and $I_2$.

\item Given $-T+Mg=Ma$ and $T-mg=ma$, solve for $a$ in terms of $M$, $m$, and $g$.

\item Sketch the following three functions on the same graph where $0\le t\le2\pi$: $x(t)=\cos t$, $x(t)=2\cos t$, and $x(t)=\sin(2t)$.

\item What are the following conversions?
\begin{enumerate}[label=(\alph*)]
\item $57~\text{g}$ to kg
\item $10~\text{m}$ to mm
\item $3~\text{mm}^2$ to $\text{m}^2$
\end{enumerate}

\item Sketch the following three functions on the same graph where $-2\le x\le2$: $y(x)=x$, $y(x)=|x|$, and $y(x)=3x$.

\item On one graph, roughly sketch the behavior of the following two functions, where $p$ and $V$ are positive: $p(V)=1/V$ and $p(V)=1/V^2$.

\item Evaluate $\sin\left(\frac{2\pi}{3}t\right)$ for $t=0,1,3$.

\item Evaluate $A\sin(kx-\omega t)$ for:
\begin{enumerate}[label=(\alph*)]
\item $x=0~\text{m}$, $t=0~\text{s}$
\item $x=1~\text{m}$, $t=0.25~\text{s}$
\end{enumerate}
Given $A=2~\text{cm}$, $k=4\pi~\text{rad/m}$, and $\omega=20\pi~\text{rad/s}$.

\item Consider the equation $pV=nRT$. If $n$ and $R$ are constant, how does $V$ change if you both double $p$ and quadruple $T$?

\item Solve the following for $T_f$ in terms of the $m$’s, $c$’s, and $T$’s:
\[
m_1c_1(T_f-T_{1i})+m_2c_2(T_f-T_{2i})=0.
\]

\item Using your previous result and $c_1=c_2$, $m_2=2m_1$, $T_{1i}=50^{\circ}\text{C}$, and $T_{2i}=80^{\circ}\text{C}$, find $T_f$.

\item Refer to the figure shown. What is $\Delta r=r_2-r_1$ in terms of $h$ and $D$ if each $r_i$ is the distance from the corresponding circle to the box? Assume the lines labeled $h$ and $D$ are perpendicular.

\end{enumerate}

\end{document}
