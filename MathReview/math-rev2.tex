\documentclass[11pt]{article}
\usepackage{amsmath,amssymb}
\usepackage{geometry}
\usepackage{enumitem}
\usepackage{siunitx}
\geometry{margin=1in}
\setlist[enumerate]{itemsep=10\baselineskip}
\title{Preliminary Math Self-Assessment\\Physics 122}
\author{Monadi}
\date{Winter 2026}

\begin{document}
\maketitle

\noindent The purpose of this self-assessment is to help you determine whether you should review some mathematical skills that are essential for success in Physics 122. Complete the problems below \textbf{without using any resources besides a scientific calculator}. You do not need to time yourself. This assessment will not be collected, but you are expected to keep a completed copy in your course binder throughout the quarter.

\medskip
\noindent Use the space below each problem to show your work clearly. Showing your work is essential for identifying errors and strengthening your understanding. Simply reading solutions afterward does not count as reviewing the material.

\medskip
\noindent An understanding of physics is \emph{not} required to answer any of the following questions; however, the mathematical tools used here will appear repeatedly throughout the course.

\begin{enumerate}[label=\arabic*.]

\item Evaluate the following expressions and express your answers in scientific notation:
\begin{enumerate}[label=(\alph*)]
\item $(3.6\times10^5)(4.2\times10^{-3})$
\item $\dfrac{7.5\times10^{-6}}{2.5\times10^{2}}$
\end{enumerate}

\item Convert the following quantities:
\begin{enumerate}[label=(\alph*)]
\item $450~\text{cm}$ to meters
\item $2.4~\text{L}$ to cubic meters
\item $0.75~\text{g/cm}^3$ to $\text{kg/m}^3$
\end{enumerate}

\item An angle is measured to be $1\frac{1}{4}$ radians.
\begin{enumerate}[label=(\alph*)]
\item Find $\sin(\theta)$ and $\cos(\theta)$.
\item Determine whether $\theta$ is closer to $90^\circ$ or $180^\circ$.
\end{enumerate}

\item Evaluate the following:
\[
\ln(5.0) + \ln(2.0) - \ln(4.0)
\]

\item Solve the following equation for $x$:
\[
3x^2 - 12x + 9 = 0.
\]

\item A quantity varies inversely with the square of $x$.
\begin{enumerate}[label=(\alph*)]
\item Write an equation expressing this relationship.
\item If the quantity has value $Q_0$ at $x=x_0$, find its value at $x=3x_0$.
\end{enumerate}

\item The pressure of a gas is given by
\[
P = \frac{nRT}{V}.
\]
\begin{enumerate}[label=(\alph*)]
\item Solve for $V$ in terms of the other variables.
\item If $P$ is doubled while $n$, $R$, and $T$ remain constant, how does $V$ change?
\end{enumerate}

% \item Evaluate the following derivative:
% \[
% \frac{d}{dx}(4x^3 - 6x).
% \]

\item A function is given by $y(t) = 2t^2 - 4t + 1$.
\begin{enumerate}[label=(\alph*)]
\item Find the value of $y$ at $t=3~\text{s}$.
\item Sketch $y(t)$ for $0 \le t \le 4~\text{s}$. Clearly label axes and units.
\end{enumerate}

\item Solve the following system of equations:
\[
\begin{cases}
2x + 3y = 13\\
4x - y = 5
\end{cases}
\]

\item The intensity of a wave is proportional to the square of its amplitude.
\begin{enumerate}[label=(\alph*)]
\item Write a proportionality relating intensity $I$ and amplitude $A$.
\item If the amplitude is tripled, by what factor does the intensity change?
\end{enumerate}

\item Evaluate the following trigonometric expression:
\[
\tan\left(\frac{3\pi}{4}\right) + \sin\left(\frac{\pi}{6}\right).
\]

\item A quantity oscillates according to
\[
x(t) = A\cos(\omega t).
\]
\begin{enumerate}[label=(\alph*)]
\item What is the maximum value of $x(t)$?
\item What is the value of $x$ when $\omega t = \pi$?
\end{enumerate}

\item Light of wavelength $\lambda$ passes through a narrow opening.
\begin{enumerate}[label=(\alph*)]
\item If $\lambda$ is doubled, how does any quantity proportional to $1/\lambda$ change?
\item How does a quantity proportional to $\lambda^2$ change?
\end{enumerate}

% \item Evaluate the following definite integral:
% \[
% \int_0^2 (3x^2 + 2x)\,dx.
% \]

\item Solve for $T_f$ in the equation
\[
m_1c_1(T_f - T_{1i}) + m_2c_2(T_f - T_{2i}) = 0.
\]

\item Using your result from the previous problem, find $T_f$ given:
\[
m_2 = 2m_1, \quad c_1 = c_2, \quad T_{1i} = 20^\circ\text{C}, \quad T_{2i} = 80^\circ\text{C}.
\]

\end{enumerate}

\end{document}
