\documentclass[12pt]{article}

\usepackage{amsmath}
\usepackage{siunitx}
\usepackage{physics}
\usepackage[margin=1in]{geometry}
\usepackage{MnSymbol,wasysym}
\usepackage{tikzsymbols}
\begin{document}

\begin{center}
\textbf{Physics Quiz -- Chapter 13: Fluids -- Winter 2026 -- Monadi  -- CP SLO} \\
\vspace{0.2cm}
\vspace{0.4cm}
Full Name:\underline{\hspace{5cm}} 
\end{center}

\[
1/2\rho v_1^2 + \rho g y_1 + P_1 = 1/2\rho v_2^2 + \rho g y_2 + P_2= Constant
\]
\[
P = P_0 + \rho g h
\]
\[
\rho = m/V
\]
\[
Q = A_1v_1 = A_2v_2
\]
\noindent


\textit{Note: Please show your complete work, write organized and clearly, draw a diagram, explain your reasoning, and write an equation 
before plugging in numbers for the full credit. A simple calculator is allowed. \smiley }

\vspace{0.4cm}
\hrule
\vspace{0.4cm}
\textbf{Problem:} \hfill (20 points)\\
A horizontal water pipe narrows from a diameter of $2\,\text{m}$ to a diameter of
$1\,\text{m}$. The speed of water flow is 3 m/s in the wider section. At each section of the pipe, a vertical tube open to the
atmosphere is attached, forming two water columns that indicate the local
pressure. The height of the water column above the narrower section is $10\,\text{m}$,
Assume steady, incompressible, non-viscous flow. Air pressure is 100.0$kPa$. 
Water density is 1000$kg/m^3$.
$g=10 m/s^2$ and $\pi = 3.14$
 
\begin{itemize}{}
    
\item[(a)] \textbf{(5 points)}   Find the velocity of water in the narrower section. 
   
\item[(b)] \textbf{(5 points)}  What is the pressure of water in the narrower section?
    
\item[(c)] \textbf{(5 points)}  Calculate the height of water column in the narrower section. 
    
\item[(d)] \textbf{(5 points)}   Explain how the height of water columns would have changed, if the there was a less input water volume rate into the pipe. 
\end{itemize}
\end{document}
