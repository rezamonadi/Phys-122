\documentclass[12pt]{article}
\usepackage{amsmath, amssymb}
\usepackage{graphicx}
\usepackage{geometry}
\geometry{margin=1in}

\title{Lab 1: Simple pendulum}
\date{}
\author{}

\begin{document}
\maketitle

\section*{Introduction}
In this experiment you investigate the behavior of a simple physical system called a pendulum, 
consisting of a mass swinging on string. Although mechanically simple, this system is important because 
it exhibits repetitive motion. The frequency of repetition depends on the physical properties of the 
system, the mass, and the string length. You will experimentally determine how the motion is related to 
system properties by taking data and performing graphical analysis to obtain a mathematical ``fit''. 
Comparison of this fit with theory results in a determination of the acceleration due to gravity.



\begin{figure}[h!]
    \begin{center}
    
    \includegraphics[width=0.25\linewidth]{Pendulum.png}
    \caption{A simple pendulum of length $L$ and mass $m$ displaced by an angle $\theta$.}
\end{center}
\end{figure}
A simple pendulum is a mass $m$ swinging on the end of a string of length $L$ whose other end is fixed, as 
shown in Figure 1. The mass swings back and forth with a period $T$. The period is the amount of time it 
takes for one complete back and forth swing. Theoretically, the period is related to the length of the 
string and to the acceleration due to gravity $g$ according to
\begin{equation}
T = 2\pi \sqrt{\frac{L}{g}} \tag{1}
\end{equation}

What we want to do in this lab is investigate the truth of this formula, and along the way get some 
practice with plotting and line-fitting.

You will see in the textbook that the formula (1) is really only applicable for small amplitude oscillations, 
and we will explore that a little later. For now, only displace the pendulum by about $\theta = 5^\circ$ before you 
release. You will also notice that formula (1) does NOT contain the mass of the pendulum. \textbf{Try changing 
the mass and see that the pendulum period is indeed independent of it.}

How best to verify formula (1)? You could just pick a value for $L$, measure the period $T$, and calculate the 
period $T$ using (1). The measured and calculated values should then agree. But that is not enough to 
verify the general applicability of (1). A better way is to vary the length of the pendulum and measure 
the period and see if the formula relating period to length works for many different situations. This 
verification could be presented as a table of numbers, but that is rather hard to read. A much better way 
is to create a graphical representation and the best graphical representation for humans to look at (if 
you can make it) is a straight line.

So, if we were to take our formula (1) and square it we would get
\begin{equation}
T^2 = \frac{4\pi^2}{g} L
\end{equation}

You can see that if we were to plot $T^2$ vs $L$ we should get a straight line because we now have an 
equation in the form $y = (\text{constant})x$ where the $y$ is $T^2$, the $x$ is $L$, and the constant, which is the 
slope of the line, is equal to $4\pi^2/g$.

\section*{Procedure to verify formula (1)}
Starting with a length $L = 0.2\ \text{m}$, and going up to $2\ \text{m}$, release the pendulum from $5^\circ$ and 
measure the period. Record the period and enter it along with the lengths into a spreadsheet. Using the 
spread sheet, calculate the square of the period and then plot it against the length. Make sure that the 
period squared is on the ``y'' axis. Now, using the fitting capability, put a straight line through your data 
and record the slope of the line (what units would it have?). If we have done everything correctly then 
the slope should equal $4\pi^2/g$. So as a check, figure out a value for the acceleration due to gravity. It 
should agree pretty closely with the value displayed on the PhET.

\section*{The pendulum at larger angles}
As mentioned above, the derivation bringing you to the formulas you’ve seen so far was done in the 
small-angle approximation. The small angle approximation is when you can say $\theta = \sin(\theta) = \tan(\theta)$ 
and where the angle must be expressed in radians. If you have never done it before, you can try it and 
see. Take a small angle like 5 degrees, express it in radians, and then take its sine. Those last two 
numbers should be 0.087266463 and 0.087155743, respectively. They differ by about 0.1\%. But try it at
bigger angles like 60 degrees. The agreement is not so good.

Similarly, formula (1) begins to break down at larger angles. There is actually a solution for the period of 
the pendulum at larger angles, but it is in terms of an infinite series. You may come across it in more 
advanced math/physics classes.

For now, let's see how good our approximate formula is. Set the pendulum length at say 1 m and make a 
series of measurements of the period for various angles, starting at 5 degrees and going up to 70 
degrees in steps of 15 degrees. Plot these with the angle on the horizontal axis and the period on the 
vertical axis. You will see that equation (1) gives you a period quite close to two seconds for this length 
of pendulum. From the graph you have produced, estimate the maximum angle you could start your 
pendulum with and for which formula (1) still gives you an answer within 5\% of the true period (period 
at very low angle).

\section*{Lab report}
Your report should include the following: 
\begin{enumerate}
    \item Include the names of your group members.
    \item Some sort of short intro about the purpose/goal of the lab. What did you accomplish?
    \item The data tables ($T$ vs $L$ and $T$ vs $\theta$) and graphs you have produced ($T$ vs. $L$, $T^2$ vs. $L$, and $T$ 
    vs $\theta$ (in degrees)). When producing your graphs make sure that the axes are correctly 
    labeled and have the correct units.
    \item Your estimate of the acceleration due to gravity and how you obtained it from the $T^2$ vs $L$ 
    graph. Show your work and calculate the percent error of $g_{measured}$ compared to $g =
    9.8\ \text{m/s}^2$.
    \item Report, referring to your second graph (or your angle which keeps your measured 
    period less than 5\% different from the calculated period), the maximum angle you 
    should use in this experiment for there to be good agreement between the approximate 
    theory (equation 1) and the measurement of period. Provide an explanation of your 
    reasoning for choosing this ``max angle''.
    \item Question 1: Are you surprised that period only depends on $L$ and $g$? Provide a short 
    explanation of why mass has no effect. 
    \item Question 2: What assumptions are we making to ensure equation 1 is correct (discuss 
    more than one). In other words, equation 1 assumes a ``simple pendulum''. What 
    simplifications need to be true to make it ``simple''.
    \item Question 3: In this lab, we ``linearize the data''. What is the point of doing this? Discuss 
    the relationship we are trying to verify in the lab.
\end{enumerate}

\end{document}
